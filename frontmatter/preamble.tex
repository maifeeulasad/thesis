\usepackage{bm}
\usepackage{geometry}
  \geometry{a4paper,inner=3cm, outer=3cm, top=3cm, bottom=3cm}
\pdfpagewidth=\paperwidth 
\pdfpageheight=\paperheight

%\usepackage[parfill]{parskip} 
% Activate to begin paragraphs with an empty (return) line, comment out the indent below if you chose the return line option.

\setlength{\parindent}{1em}  % Sets the length of the paragraph indent. Current setup has a an indent. Disable this if you activate the return line above.

% Double or one and a half spacing.
%\usepackage{setspace}
% \singlespacing

  
\usepackage{graphicx}
  \DeclareGraphicsRule{.tif}{png}{.jpg}{`convert #1 `dirname #1`/`basename #1 .tif`.png}
 \graphicspath{{figures/}}
% Graphics. Remove me and you won't have any figures, and that would be very boring.

%\usepackage[usenames,dvipsnames,svgnames,table]{xcolor}
% Adds the ability to make coloured text and lines throughout the document. See documentation for xcolor.

%-------------------- Tables, figures and captions
\usepackage[font={small},labelfont={bf},margin=4ex]{caption}
% Makes bold labeled and smaller font captions. Must be loaded before the longtable package to avoid conflicts! 

\usepackage{longtable}
% Long tables (more than one page). Different headers and footers for beginning and end pages, etc.

\usepackage{afterpage}
% Make a longtable start on the next clear page, but fills the previous one with text first (no random gaps in the text-from long tables anymore! Man, the day I discovered this...)

\usepackage{booktabs}
% Nice looking tables and lines in tables

\usepackage{multirow}
% Entries in tables over multiple rows

\usepackage{lscape}
% Pages in landscape

\usepackage{pdflscape}
% Landscape pages also rotated in the pdf

\usepackage{wrapfig}
% Allows figures that don't take up the entire width of the page, wrapping the text around the figure

%\usepackage[position=top,singlelinecheck=false,captionskip=4pt]{subfig} 
% Multiple figures in an individual figure. Fig. 1 a, b, c, etc. each with, or without, it's own individual caption, and with a global caption for all sub figures.

%-------------------- Special symbols and fonts
\usepackage{amssymb}
% Maths symbols

%-------------------- Document sections, headers, footers, and bibliography
\usepackage{fancyhdr}

%-------------------- Bibliography

\usepackage[style=chicago-authordate,strict,backend=bibtex8,babel=other,bibencoding=inputenc]{biblatex}

%\usepackage{biblatex-chicago}


%\usepackage[notes,backend=biber]{biblatex-chicago}
%\usepackage[strict,backend=bibtex8,babel=other,bibencoding=inputenc]{biblatex-chicago}

%-------------------- Hyperlinks in your document.
\usepackage{hyperref}
\hypersetup{
	colorlinks   = true,
	allcolors    = black
}
% The hyperref package allows you to have clickable links in your pdf. It also allows you to have the mailto link associated with your name. It can be  a bit finicky, so load it last.

%-------------------- Command renewals, New commands etc.
\renewcommand{\thefootnote}{\arabic{footnote}}

%-------------------- Add your own packages here


\usepackage[utf8]{inputenc}
\usepackage{spverbatim}
\usepackage{pythonhighlight}
\usepackage{graphicx} 
\usepackage{caption,subcaption}
\usepackage{pgfplots}
\usepackage{svg}
\usepackage{pgf-pie}  
\usepackage{tikz}
\usetikzlibrary{shapes.geometric,arrows} 

% \usepackage[numbers]{natbib}
% \setcitestyle{numbers}


\usetikzlibrary{positioning}


\newcommand{\R}{\mathbb{R}}

\tikzset{%
  every neuron/.style={
    circle,
    draw,
    minimum size=1cm
  },
  neuron missing/.style={
    draw=none, 
    scale=4,
    text height=0.333cm,
    execute at begin node=\color{black}$\vdots$
  },
}





\tikzstyle{block} = [rectangle, draw, fill=yellow!50,   
    text width=4.5em, text centered, rounded corners, minimum height=4em] 
\tikzstyle{line} = [draw, -latex']  
\tikzstyle{cloud} = [draw, ellipse,text width= 2.9em, fill=red!50, node distance=2cm, minimum height=3em]  
 \tikzstyle{io} = [trapezium, draw, trapezium right angle=110, rounded corners, fill=red!20, node distance=1.9cm, minimum height=2.9em]    
 
 
\usepackage{amssymb}
\usepackage{bm}
\usepackage{tikz}
\usetikzlibrary{arrows.meta}
\usetikzlibrary{decorations.pathreplacing}




\definecolor{c1}{HTML}{DCDCDC}
\definecolor{c2}{HTML}{FFFFFF}
\definecolor{c3}{HTML}{DCDCDC}
\definecolor{c4}{HTML}{FFFFFF}
\definecolor{c5}{HTML}{DCDCDC}
\definecolor{c6}{HTML}{FFFFFF}
\newcommand*{\xMin}{0}%
\newcommand*{\xMax}{6}%
\newcommand*{\yMin}{0}%
\newcommand*{\yMax}{4}%

\newcommand*{\xMinR}{9.5}%
\newcommand*{\xMaxR}{12.5}%
\newcommand*{\yMinR}{1}%
\newcommand*{\yMaxR}{3}%



\newcommand{\distance}{6}
\newcommand{\xup}{3.5}
\newcommand{\yup}{6}
\newcommand{\upsizex}{1}
\newcommand{\upsizey}{2}
\newcommand{\upshift}{3/4*\upsizey}
\newcommand{\distancedots}{1}




\definecolor{rtd}{HTML}{000000}
\definecolor{dgd}{HTML}{000000}



\usetikzlibrary{patterns}
\usepgfplotslibrary{groupplots}

\usepackage{setspace}
\setstretch{1}


\usepackage{nomencl}



\usetikzlibrary{shapes,arrows,calc}

\newcommand{\circular}[2]{
\node (0) at (0,0){};
\xdef\Radious{#2}
\foreach \nodetext[count=\i] in {#1} {
\xdef\totalIt{\i}
}
\foreach \nodetext[count=\i] in {#1} {
\pgfmathsetmacro\myprev{int{\i-1}}
\pgfmathsetmacro\curxdispl{cos(90-360/\totalIt*(\i-1))*\Radious}
\pgfmathsetmacro\curydispl{sin(90-360/\totalIt*(\i-1))*\Radious}
\path  ($(\curxdispl cm,\curydispl cm)$) node[block,anchor=center] (\i) {\nodetext};
}

\foreach \nodetext[count=\i] in {#1}{
\pgfmathsetmacro\mynext{\ifnum\i<\totalIt int(\i+1)\else 1\fi}
\path[draw,-latex] (\i) to(\mynext);
}
}

\usepackage{lastpage}



\renewcommand{\headrulewidth}{0pt}
\fancyhead{}
\pagestyle{fancy}
